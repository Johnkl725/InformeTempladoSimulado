\documentclass{article}
\usepackage[utf8]{inputenc} % Paquete para definir la codificación de entrada
\usepackage[T1]{fontenc} % Paquete para definir la codificación de salida
\usepackage{tocbibind} % Paquete para incluir el índice en el contenido
\usepackage{listings} % Paquete para incluir código fuente
\usepackage{xcolor} % Paquete para definir colores

\begin{document}

\title{
Facultad de Ingeniería de Sistemas e Informática\\
E.A.P de Ingeniería de Sistemas e Informática\\
\vspace{1cm}
TRABAJO GRUPAL\\
\vspace{1cm}
GRUPO N°3\\
\vspace{1cm}
Curso: Investigación Operativa\\
Docente: Cubas Becerra, Richard Javier\\
\vspace{1cm}
Integrantes:\\
Aldana Chipana, Mauricio \hfill [22200164]\\
Castillo Reupo, John \hfill [22200117]\\
Escribas Alan, Daniel \hfill [22200057]\\
Espíritu Unsihuay, Erika Milagros \hfill [22200170]\\
Valdiviezo Goicochea, Wisneria \hfill [22200217]\\
\vspace{1cm}
Lima, Perú\\
2024
}
\date{}

\maketitle
\newpage % Comando para empezar una nueva página después del título

\tableofcontents % Comando para generar el índice
\newpage % Comando para empezar una nueva página después del índice

\section{Introducción al Templado Simulado}
El templado simulado es un algoritmo de optimización inspirado en el proceso de enfriamiento de metales. Este método es utilizado para encontrar soluciones aproximadas a problemas de optimización complejos.

\section{Funcionamiento del Algoritmo}
\section{Inspiración en el recocido metálico}

\subsection{Similitudes entre el enfriamiento progresivo de metales y el comportamiento del algoritmo}

El \textit{templado simulado} toma su fundamento del proceso de recocido metálico (\textit{annealing}), utilizado desde la antigüedad en la metalurgia para alterar las propiedades físicas y mecánicas de los materiales. Este procedimiento inspiró el diseño del algoritmo para resolver problemas de optimización combinatoria.

\subsection{Recocido en metalurgia}

\subsubsection{Proceso físico:}
\begin{itemize}
    \item Consiste en calentar un metal a temperaturas altas y luego enfriarlo lentamente.
    \item Durante la fase de calentamiento, las partículas del material adquieren un estado de alta energía, reorganizándose de manera aleatoria dentro de su estructura cristalina.
    \item Al enfriar de forma progresiva, las partículas se asientan en configuraciones más estables y óptimas, reduciendo la energía total del sistema.
\end{itemize}

\subsubsection{Objetivo del recocido:}
\begin{itemize}
    \item Incrementar la resistencia del material.
    \item Reducir tensiones internas para prevenir fracturas o deformaciones.
    \item Mejorar propiedades mecánicas, como la dureza y la ductilidad.
\end{itemize}

\section{Analogía con el algoritmo de templado simulado}

\subsection{Sistema térmico vs. sistema de optimización:}
\begin{itemize}
    \item En el recocido metálico, el sistema físico es el material, y la \textit{energía} representa el nivel de tensión en su estructura.
    \item En el algoritmo, el sistema corresponde a un espacio de soluciones, y la \textit{energía} se traduce como el costo asociado a una solución.
\end{itemize}

\subsection{Fases del algoritmo y su relación con el recocido:}
\begin{itemize}
    \item \textbf{Calentamiento inicial:} Al igual que el recocido físico, el algoritmo comienza con una \textit{temperatura} alta, lo que permite explorar soluciones aleatorias con gran libertad.
    \item \textbf{Enfriamiento gradual:} A medida que la temperatura disminuye, el sistema se orienta hacia soluciones más estables, rechazando cambios drásticos pero aceptando soluciones subóptimas bajo ciertas condiciones.
\end{itemize}

\subsection{Puntos clave de la analogía}

\subsubsection{Estados del sistema:}
\begin{itemize}
    \item Cada posible configuración del material metálico en el recocido corresponde a una solución en el espacio de búsqueda del algoritmo.
\end{itemize}

\subsubsection{Enfriamiento controlado:}
\begin{itemize}
    \item Si el recocido físico se enfría demasiado rápido, el material puede cristalizar en estados no óptimos, resultando en fragilidad o imperfecciones.
    \item De forma análoga, en el algoritmo, un enfriamiento brusco puede conducir a mínimos locales en lugar del óptimo global.
\end{itemize}

\section{Implicaciones teóricas}

El éxito del algoritmo de templado simulado radica en el balance adecuado entre la exploración inicial del espacio de búsqueda (alta temperatura) y la explotación posterior de las mejores soluciones (baja temperatura). Este enfoque, inspirado directamente en el recocido metálico, proporciona una base robusta para su efectividad en problemas de optimización.

\section{Funcionamiento del Algoritmo de Recocido Simulado}

El algoritmo de recocido simulado se basa en los principios de la termodinámica aplicados al recocido de metales, en donde se busca minimizar la energía de un sistema al enfriarlo lentamente. Este procedimiento se traduce al ámbito de optimización combinatoria como un proceso iterativo para encontrar el óptimo global en un espacio de búsqueda.

\subsection{Etapas del Algoritmo}
\begin{enumerate}
    \item \textbf{Inicialización:}
    \begin{itemize}
        \item Se define una \textit{solución inicial} ($S_0$) de manera aleatoria en el espacio de búsqueda.
        \item Se establece una \textit{temperatura inicial} ($T_0$), que controla la probabilidad de aceptar soluciones peores.
    \end{itemize}
    
    \item \textbf{Ciclo de Evaluación y Generación:}
    \begin{itemize}
        \item Se genera una \textit{solución vecina} ($S'$) a partir de $S_0$ mediante una ligera alteración (perturbación).
        \item Se evalúan ambas soluciones usando la función objetivo $f(x)$ para determinar el cambio en el costo ($\Delta f = f(S') - f(S_0)$).
    \end{itemize}
    
    \item \textbf{Criterio de Aceptación:}
    \begin{itemize}
        \item Si $f(S') < f(S_0)$, la solución vecina mejora la actual y se acepta.
        \item Si $f(S') > f(S_0)$, se evalúa la probabilidad de aceptar la solución subóptima utilizando el \textit{criterio de Boltzmann:}
        \[
        P = \exp\left(-\frac{\Delta f}{T}\right)
        \]
        Un número aleatorio $r \in [0, 1]$ se compara con $P$. Si $P > r$, se acepta la nueva solución.
    \end{itemize}
    
    \item \textbf{Condición de Parada:}
    \begin{itemize}
        \item El algoritmo finaliza cuando se alcanza una temperatura mínima ($T_f$) o un número máximo de iteraciones.
    \end{itemize}
\end{enumerate}

\subsection{Pseudocódigo del Algoritmo}
El pseudocódigo actualizado describe el funcionamiento del algoritmo de manera estructurada:

\begin{verbatim}
Función Simulated_Annealing()
    T := T0 // Temperatura inicial
    S := Generar_Solución_Inicial() // Solución inicial
    Mientras (T > Tf y no alcanzar condición de parada)
        S' := Generar_Solución_Vecina(S) // Generar nueva solución
        Si (f(S') < f(S)) Entonces
            S := S' // Aceptar la nueva solución
        Sino
            r := Aleatorio(0,1) // Generar número aleatorio
            Si (r < exp(-Δf / T)) Entonces
                S := S' // Aceptar con probabilidad P
            Fin_Si
        Fin_Si
        T := Reducir_Temperatura(T) // Aplicar enfriamiento
    Fin_Mientras
    Retornar(S) // Devolver la mejor solución encontrada
Fin_Función
\end{verbatim}

\subsection{Análisis del Criterio de Boltzmann}

El \textit{criterio de Boltzmann} es el núcleo probabilístico que permite aceptar soluciones peores bajo ciertas condiciones, lo que ayuda a escapar de mínimos locales. Se analiza mediante las siguientes variables:

\begin{enumerate}
    \item \textbf{Impacto de $\Delta f$:}
    \begin{itemize}
        \item Si $\Delta f$ es pequeño ($S'$ es ligeramente peor que $S_0$), $P$ es alto, favoreciendo la aceptación.
        \item Si $\Delta f$ es grande ($S'$ es significativamente peor), $P$ es bajo, reduciendo la probabilidad de aceptar.
    \end{itemize}
    
    \item \textbf{Efecto de la Temperatura ($T$):}
    \begin{itemize}
        \item A altas temperaturas, $P$ es mayor, lo que fomenta saltos exploratorios amplios.
        \item A bajas temperaturas, $P$ disminuye, restringiendo la aceptación a soluciones cercanas y afinando la búsqueda.
    \end{itemize}
    
    \item \textbf{Relación entre Variables:}
    \begin{itemize}
        \item La probabilidad de aceptar soluciones peores disminuye exponencialmente con el tiempo, lo que asegura la convergencia del algoritmo hacia un óptimo global.
    \end{itemize}
\end{enumerate}

\section{Escapar de Mínimos Locales}

Un desafío recurrente en la optimización es quedar atrapado en \textit{mínimos locales}: soluciones que son mejores dentro de una pequeña región, pero no en todo el espacio de búsqueda. El templado simulado resuelve este problema de tres formas:

\subsection{1. Aceptación de Soluciones Subóptimas:}
\begin{itemize}
    \item A diferencia de otros algoritmos, el templado simulado permite aceptar soluciones que empeoran temporalmente la calidad de la solución actual.
    \item En las primeras etapas, cuando la \textit{temperatura} es alta, la probabilidad de aceptar estas soluciones es elevada, lo que permite al algoritmo explorar más ampliamente.
    \item Este mecanismo ayuda a escapar de los mínimos locales tempranos, explorando otras regiones del espacio de búsqueda.
\end{itemize}

\subsection{2. Exploración y Explotación Balanceada:}
\begin{itemize}
    \item Al inicio (alta temperatura), el algoritmo fomenta la \textit{exploración} de soluciones diversas, moviéndose libremente por el espacio.
    \item Conforme la temperatura disminuye, la búsqueda se vuelve más \textit{selectiva}, enfocándose en mejorar la solución encontrada (\textit{explotación}).
\end{itemize}

\subsection{3. Simulación de un Proceso Natural:}
\begin{itemize}
    \item Este comportamiento emula el recocido de metales: al enfriarse lentamente, las partículas del material se reorganizan hasta alcanzar una estructura estable y óptima.
    \item En el algoritmo, las soluciones se \textit{estabilizan} conforme la temperatura disminuye, convergiendo hacia un óptimo global.
\end{itemize}

\section{Justificación Matemática de su Flexibilidad}

El templado simulado está basado en principios matemáticos sólidos, como las \textit{cadenas de Markov} y las \textit{distribuciones de Boltzmann}.

\subsection{1. Cadenas de Markov:}
\begin{itemize}
    \item Cada iteración del algoritmo representa un \textit{estado} en una cadena de Markov, donde las transiciones entre estados (soluciones) dependen únicamente del estado actual.
    \item Esto garantiza que las decisiones de aceptación o rechazo de soluciones sean consistentes y controladas, proporcionando un modelo matemático predecible.
\end{itemize}

\subsection{2. Distribuciones de Probabilidad:}
\begin{itemize}
    \item La probabilidad de aceptar soluciones peores está definida por la fórmula:
    \[
    P = \exp\left(-\frac{\Delta f}{T}\right)
    \]
    \item Donde:
    \begin{itemize}
        \item $\Delta f$: Diferencia en el valor de la función objetivo entre la solución actual y la nueva.
        \item $T$: Temperatura actual.
    \end{itemize}
    \item En etapas avanzadas, el algoritmo favorece soluciones de menor costo, asegurando un refinamiento progresivo hacia el óptimo global.
\end{itemize}

\subsection{3. Modelos de Enfriamiento Personalizables:}
\begin{itemize}
    \item Diferentes esquemas de enfriamiento permiten adaptar el algoritmo a las necesidades de cada problema:
    \begin{itemize}
        \item \textbf{Geométrico:} Fácil de implementar y eficiente.
        \item \textbf{Logarítmico:} Más lento, pero garantiza la convergencia matemática al óptimo global.
        \item \textbf{Lundy y Mees:} Ajusta la reducción de temperatura para equilibrar exploración y precisión.
    \end{itemize}
\end{itemize}

\section{Justificación Matemática de su Convergencia}

El diseño probabilístico del algoritmo garantiza que, con un enfriamiento suficientemente lento, converge al \textit{óptimo global}.

\subsection{1. Convergencia Teórica:}
\begin{itemize}
    \item Al reducir la temperatura de forma progresiva, el sistema tiene tiempo suficiente para explorar y refinar las soluciones, evitando quedarse atrapado en mínimos locales.
\end{itemize}

\subsection{2. Modelos de Enfriamiento:}
\begin{itemize}
    \item El modelo \textit{logarítmico} garantiza la convergencia matemática, aunque a un ritmo lento.
    \item El modelo \textit{geométrico} equilibra velocidad y efectividad, siendo práctico para muchas aplicaciones.
\end{itemize}

\section{Beneficios Adicionales}

\subsection{1. Resiliencia a Problemas NP-Duros:}
\begin{itemize}
    \item Es aplicable a problemas donde los enfoques deterministas son inviables debido a la complejidad del espacio de búsqueda.
\end{itemize}

\subsection{2. Facilidad de Implementación:}
\begin{itemize}
    \item Aunque los fundamentos teóricos son complejos, el algoritmo en sí es simple de implementar. Solo requiere ajustar unos pocos parámetros clave:
    \begin{itemize}
        \item Temperatura inicial ($T_0$).
        \item Tasa de enfriamiento ($\alpha$).
        \item Función objetivo ($f(x)$).
    \end{itemize}
\end{itemize}


\section{Parámetros del Algoritmo}
Los parámetros clave del algoritmo de templado simulado incluyen:
\begin{itemize}
    \item Temperatura inicial (\( T_0 \))
    \item Tasa de enfriamiento (\( \alpha \))
    \item Número de iteraciones por temperatura
\end{itemize}

Estos parámetros deben ser ajustados cuidadosamente para obtener un buen rendimiento del algoritmo.

\section{Análisis y Perspectivas}
\subsection{Ventajas}
\begin{itemize}
    \item Capacidad para escapar de óptimos locales.
    \item Flexibilidad para ser aplicado a una amplia variedad de problemas.
\end{itemize}

\subsection{Limitaciones}
\begin{itemize}
    \item Sensibilidad a la elección de parámetros.
    \item Puede ser computacionalmente costoso para problemas muy grandes.
\end{itemize}

\section{Caso Aplicativo}
Para ilustrar el uso del templado simulado, consideremos un problema de optimización en la investigación operativa, como la optimización de rutas de transporte. El objetivo es encontrar rutas eficientes que minimicen el costo total de transporte, considerando restricciones como la capacidad de los vehículos y las ventanas de tiempo de entrega.

El problema de optimización de rutas de transporte (Vehicle Routing Problem, VRP) es un problema clásico en la investigación operativa. En este problema, un conjunto de vehículos debe recoger y entregar bienes a un conjunto de clientes, minimizando el costo total de transporte. Las restricciones incluyen la capacidad de los vehículos y las ventanas de tiempo en las que las entregas deben realizarse.

El algoritmo de templado simulado puede ser utilizado para encontrar soluciones aproximadas a este problema. A continuación, se presenta una implementación en Python del algoritmo de templado simulado para resolver el VRP.

\textbf{Implementación en Python}

\lstset{language=Python, 
        basicstyle=\ttfamily\footnotesize, 
        keywordstyle=\color{blue}, 
        stringstyle=\color{red}, 
        commentstyle=\color{green}, 
        showstringspaces=false, 
        numbers=left, 
        numberstyle=\tiny\color{gray}, 
        breaklines=true, 
        frame=single, 
        captionpos=b}

\begin{lstlisting}
import math
import random

#Funcion de costo (distancia total de la ruta)
def calculate_cost(route, distance_matrix):
    cost = 0
    for i in range(len(route) - 1):
        cost += distance_matrix[route[i]][route[i + 1]]
    cost += distance_matrix[route[-1]][route[0]]  # Regresar al punto de inicio
    return cost

# Generar una solucion vecina
def generate_neighbor(route):
    new_route = route[:]
    i, j = random.sample(range(len(route)), 2)
    new_route[i], new_route[j] = new_route[j], new_route[i]
    return new_route

# Algoritmo de templado simulado
def simulated_annealing(distance_matrix, initial_temp, cooling_rate, num_iterations):
    current_route = list(range(len(distance_matrix)))
    random.shuffle(current_route)
    current_cost = calculate_cost(current_route, distance_matrix)
    best_route = current_route[:]
    best_cost = current_cost
    temperature = initial_temp

    for _ in range(num_iterations):
        new_route = generate_neighbor(current_route)
        new_cost = calculate_cost(new_route, distance_matrix)
        delta_cost = new_cost - current_cost

        if delta_cost < 0 or random.random() < math.exp(-delta_cost / temperature):
            current_route = new_route
            current_cost = new_cost

            if current_cost < best_cost:
                best_route = current_route
                best_cost = current_cost

        temperature *= cooling_rate

    return best_route, best_cost

# Ejemplo de uso
distance_matrix = [
    [0, 2, 9, 10],
    [1, 0, 6, 4],
    [15, 7, 0, 8],
    [6, 3, 12, 0]
]

initial_temp = 1000
cooling_rate = 0.99
num_iterations = 10000

best_route, best_cost = simulated_annealing(distance_matrix, initial_temp, cooling_rate, num_iterations)
print("Mejor ruta:", best_route)
print("Costo de la mejor ruta:", best_cost)
\end{lstlisting}

Este código proporciona una base para implementar el algoritmo de templado simulado en Python. Puedes adaptarlo y expandirlo según tus necesidades específicas.
\subsection{El Problema del Viajante de Comercio (Traveling Salesman Problem, TSP)}
\textbf{¿En qué consiste?}

El problema del Viajante de Comercio es un clásico problema de optimización combinatoria que se define así:

Dado un conjunto de n ciudades y las distancias entre cada par de ciudades, el objetivo es encontrar la ruta más corta que:
\begin{itemize}
    \item Visite cada ciudad exactamente una vez.
    \item Regrese a la ciudad de origen.
\end{itemize}

\textbf{Ejemplo}

Imagina que un vendedor necesita visitar 5 ciudades:

\begin{itemize}
    \item Ciudades: A, B, C, D, E.
    \item Distancias: Una tabla o un mapa que muestra la distancia entre cada par de ciudades.
\end{itemize}

El viajante debe encontrar la ruta óptima que minimice la distancia total, visitando todas las ciudades y regresando al punto de partida.

\textbf{Complejidad del Problema}

El problema del Viajante es NP-completo, lo que significa que no existe (hasta ahora) un algoritmo eficiente que resuelva todas las instancias del problema en tiempo polinómico. La cantidad de posibles rutas crece factorialmente (\(n!\)) con el número de ciudades:

\begin{itemize}
    \item Para 5 ciudades: \(5! = 120\) rutas posibles.
    \item Para 10 ciudades: \(10! = 3,628,800\) rutas posibles.
\end{itemize}

Por lo tanto, para un gran número de ciudades, no es factible calcular exhaustivamente todas las rutas posibles.

\textbf{Resolviendo el Problema con Templado Simulado}

El Templado Simulado (Simulated Annealing, SA) es una técnica heurística que encuentra soluciones cercanas a la óptima sin necesidad de explorar todas las rutas posibles. Es ideal para problemas como el TSP debido a su capacidad de escapar de mínimos locales durante la optimización.

\textbf{Cómo funciona el Templado Simulado para el TSP}

\begin{itemize}
    \item \textbf{Representación de la Ruta:} Una ruta se representa como una permutación de las ciudades, por ejemplo:
    \begin{itemize}
        \item Ruta inicial: \(A \rightarrow B \rightarrow C \rightarrow D \rightarrow E \rightarrow A\).
    \end{itemize}
    \item \textbf{Función de Energía:} La energía es el costo total de la ruta, es decir, la suma de las distancias entre las ciudades en el orden actual.
    \begin{itemize}
        \item Costo de la ruta = Distancia(\(A \rightarrow B\)) + Distancia(\(B \rightarrow C\)) + ... + Distancia(\(E \rightarrow A\)).
    \end{itemize}
    \item \textbf{Generación de Vecinos:} Un vecino se genera modificando ligeramente la ruta actual:
    \begin{itemize}
        \item Intercambiando dos ciudades.
        \item Revirtiendo el orden de una subsección de la ruta.
    \end{itemize}
    \item \textbf{Aceptación de Vecinos:} Si el vecino tiene un costo menor (\(\Delta E < 0\)), se acepta como la nueva ruta. Si el vecino tiene un costo mayor (\(\Delta E > 0\)), se acepta con una probabilidad que depende de la diferencia de costo (\(\Delta E\)) y la temperatura (\(T\)):
    \begin{equation}
    P = e^{-\Delta E / T}
    \end{equation}
    Esto permite al algoritmo explorar soluciones subóptimas inicialmente para evitar quedar atrapado en mínimos locales.
    \item \textbf{Reducción de la Temperatura:} La temperatura (\(T\)) se reduce gradualmente después de cada iteración (enfriamiento).
    \item \textbf{Finalización:} El algoritmo termina cuando la temperatura alcanza un valor mínimo (\(T_{min}\)), devolviendo la mejor ruta encontrada.
\end{itemize}



\end{document}

Este código proporciona una base para implementar el algoritmo de templado simulado en Python. Puedes adaptarlo y expandirlo según tus necesidades específicas.


\end{document}